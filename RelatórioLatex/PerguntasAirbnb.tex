\documentclass[a4paper]{article}
%\usepackage[utf8]{inputenc}
%\usepackage[english]{babel}
\usepackage[main=portuguese,english]{babel}
\usepackage[utf8x]{inputenc}
\usepackage[T1]{fontenc}
%\usepackage{datetime}
\usepackage[ddmmyyyy]{datetime}
\usepackage{array}
\usepackage{xcolor}
\usepackage{listings}
\usepackage{verbatim}
\usepackage[a4paper]{geometry}
\usepackage{amsmath}
\usepackage{amssymb}
\usepackage{float}
\usepackage{natbib}
\usepackage{graphicx}
\usepackage{textcomp}
\usepackage[colorinlistoftodos]{todonotes}
\usepackage[colorlinks=true, allcolors=blue]{hyperref}
%\usepackage[colorlinks=true, linkcolor=blue, urlcolor=red]{hyperref}

\lstset{basicstyle=\ttfamily, showstringspaces=false, commentstyle=\color{red}, keywordstyle=\color{blue}}

\title{Questionário \textit{Airbnb} - ES}
\author{Cleyde Varela - 21684 \\ Ricardo Sequeira - 21905}
\date{Maio 2022}

\begin{document}

\maketitle

\section{Perguntas}

\subsection{Onde ouviu falar da \textit{Airbnb}?}
A primeira vez que ouvi falar do \textit{Airbnb} foi quando estava à procura de um hostel barato para passar férias com amigos, mas não estava a conseguir encontrar nenhum que correspondesse às nossas espectavas, então a minha prima recomendou que procurasse antes uma casa para alugar no \textit{site} \textit{Airbnb}, e não só consegui encontrar casas localizadas em melhores sítios como a preços mais razoáveis que a competição.

\subsection{Qual foi a primeira impressão acerca da ideia do serviço?}
A minha primeira impressão ao usar o serviço foi bastante positiva pois tudo correspondeu e até superou as nossas expectativas do mesmo.

\subsection{O que difere a \textit{Airbnb} no seu ponto de vista dos sistemas semelhantes?}
O diferenciador primário da \textit{Airbnb} dos seus competidores é na minha opinião que os imóveis que lá encontramos para alugar são de outros utilizadores do \textit{site} e não empresas que são donas dos mesmos, como por exemplo no \textit{booking.com} em que na sua maioria apenas existem hotéis e pensões. 

\subsection{Como classifica a experiência/as que teve com a\textit{Airbnb} em termos de tempo de busca, preços, nível de apoio, facilidade na utilização do sistema, eficiência…?}
A minha experiencia com o \textit{site} tem sendo sempre até agora bastante positiva, as ferramentas de busca são bem optimizadas e permitem uma pesquisa bastante rápida, tal como na altura do pagamento que também é sempre rápido e intuitivo, mostrando sempre todos os custos e sem taxas adicionais quando se chega ao fim da reserva.

\subsection{Como é efetuado o registo no \textit{site}?}
O registo no \textit{site} pode ser feito de várias maneiras, através de apenas um numero de telefone, através de criação de uma conta com um enderenço de email, ou login através da conta de \textit{facebook}, \textit{google} ou \textit{apple}.

\subsection{Que dados são pedidos aquando da criação da conta?}
É pedido ao utilizador, o primeiro e ultimo nome, data de nascimento, e no caso de não estar a entrar com uma conta existente de outro serviço (\textit{google}, \textit{apple}, \textit{facebook}) que crie uma \textit{password}.

\subsection{Depois de inseridos esses dados é necessário mais algumas informações?}
Sim depois de inserido esses dados é necessário concordar com a politica de privacidade do \textit{site}, e inserir um número de telefone válido, e pedida uma foto para anexar ao perfil, estes dados são facultativos na criação de conta pelo que pode ser adicionados mais tarde, o número de telefone torna-se obrigatório no caso de querermos alugar um alojamento para que o arrendatário tenha algum modo de contacto \textit{offline}.

\subsection{Como funciona o processo de alugar um alojamento?}
Primeiro começamos com o procura do sitio desejando, a barra de pesquisa do airbnb permite refinar a pesquisa selecionando, a localização, a data de \textit{check-in} e de \textit{check-out} e quantos hospedes ficarão hospedados.
Depois de feita a pesquisa com esses termos e escolhido a habitação onde desejamos ficar hospedados e depois de verificar os preços e as caraterísticas da habitação, é só clicar em reservar.
Depois somos levado para uma página onde temos que inserir o numero de telefone (caso não tinha sido inserido aquando do registo inicial) e inserir os dados de pagamento.

\subsection{Que tipos de pagamento são aceites?}
São aceite pagamentos com cartão de crédito (\textit{visa}, \textit{amex}, \textit{mastercard}) bem como \textit{paypal} e \textit{google pay}.

\subsection{O pagamento é feito depois na integra ou permitem pagamentos parcelados?}
Dependendo do custo do alojamento pode ser permitido pagar uma parte aquando da reserva e o restante uma semana antes da data de chegada, a percentagem do pagamento adiantado varia de alojamento para alojamento pode tanto ser 20\% aquando do pagamento, como pode ser 60\%, alojamentos com valores mais baixos apenas oferecem pagamento na integra aquando da reserva.

\subsection{E como funciona no caso de cancelarmos a estadia?}
As politicas de cancelamento são diferentes de alojamento para alojamento, podem ter cancelamento como devolução de 100\% do custo 48 horas antes do dia de chegada, a partir desse dia 50\% de custo menos a taxa de serviço, ou então em certos alojamento podemos pagar um taxa extra para obter as mesmas condições ditas anteriormente.

\subsection{E se for necessário comunicar alguma coisa ao hospede como podemos faze-lo?}
A comunicação com o anfitrião pode ser feita de varias maneiras, ambos o \textit{site} e a aplicação do \textit{Airbnb} possuem um \textit{chat} direto com o anfitrião logo após ter sido feita a reserva, ficamos também com acesso ao número de telefone do anfitrião e temos a possibilidade de lhe pedir o \textit{email} ou qualquer outro meio de comunicação que achemos mais conveniente (que poderá ou não ser disponibilizado dependendo da boa vontade do anfitrião).

\subsection{E quando chegar o dia do \textit{check-in} qual é o processo seguido para obter aceso ao alojamento?}
O processo varia de anfitrião para anfitrião, mas normalmente antes do dia do \textit{check-in} chegar é acordada uma hora e uma localização (a do alojamento normalmente), sendo o procedimento o mesmo no \textit{check-out}.
Pode também acontecer caso o alojamento ser num prédio com porteiro a chave ser deixado com o mesmo pelo anfitrião.

\end{document}
