\documentclass[a4paper]{article}
%\usepackage[utf8]{inputenc}
%\usepackage[english]{babel}
\usepackage[portuguese]{babel}
\usepackage[utf8x]{inputenc}
\usepackage[T1]{fontenc}
%\usepackage{datetime}
\usepackage[ddmmyyyy]{datetime}
\usepackage{array}
\usepackage{xcolor}
\usepackage{listings}
\usepackage{verbatim}
\usepackage[a4paper]{geometry}
\usepackage{amsmath}
\usepackage{amssymb}
\usepackage{natbib}
\usepackage{graphicx}
\usepackage[colorinlistoftodos]{todonotes}
\usepackage[colorlinks=true, allcolors=blue]{hyperref}
%\usepackage[colorlinks=true, linkcolor=blue, urlcolor=red]{hyperref}

\lstset{basicstyle=\ttfamily, showstringspaces=false, commentstyle=\color{red}, keywordstyle=\color{blue}}

\title{\textit{Engenharia de Software}\\Ponto de Situação Aplicação AirBnB} % Doc name
\author{Cleyde Varela - 21684 \\ Ricardo Sequeira - 21905} % Doc's author/s
\date{\today}

\begin{document}

\maketitle
\begin{figure}[!t]
    \includegraphics[scale=0.2]{estig.png}
\end{figure}
\newpage
\tableofcontents
\newpage

\section{Introdução}
Neste relatório iremos falar sobre as ferramentas e técnicas que propomos ser usadas na criação da aplicação AirBnB.


\newpage
\section{Conclusão}

\section{Webgrafia}
%\href{https://www.cs.princeton.edu/~wayne/kleinberg-tardos/}{Princeton's Algorithm Design Class}\\
%\href{https://www.mathworks.com/matlabcentral/answers/index}{Fórum de ajuda do MATLAB}
\bibliographystyle{plain}
\bibliography{references}
\end{document}