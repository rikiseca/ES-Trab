\documentclass[a4paper]{article}
%\usepackage[utf8]{inputenc}
%\usepackage[english]{babel}
\usepackage[portuguese]{babel}
\usepackage[utf8x]{inputenc}
\usepackage[T1]{fontenc}
%\usepackage{datetime}
\usepackage[ddmmyyyy]{datetime}
\usepackage{array}
\usepackage{xcolor}
\usepackage{listings}
\usepackage{verbatim}
\usepackage[a4paper]{geometry}
\usepackage{amsmath}
\usepackage{amssymb}
\usepackage{natbib}
\usepackage{graphicx}
\usepackage[colorinlistoftodos]{todonotes}
\usepackage[colorlinks=true, allcolors=blue]{hyperref}
%\usepackage[colorlinks=true, linkcolor=blue, urlcolor=red]{hyperref}

\lstset{basicstyle=\ttfamily, showstringspaces=false, commentstyle=\color{red}, keywordstyle=\color{blue}}

\title{\textit{Engenharia de Software}\\Ponto de Situação Aplicação AirBnB} % Doc name
\author{Cleyde Varela - 21684 \\ Ricardo Sequeira - 21905} % Doc's author/s
\date{\today}

\begin{document}

\maketitle
\begin{figure}[!t]
    \includegraphics[scale=0.2]{estig.png}
\end{figure}
\newpage
\tableofcontents
\newpage

\section{Introdução}
Neste relatório iremos falar sobre as ferramentas e técnicas que propomos ser usadas na criação da aplicação AirBnB.

\newpage

\section{Softwares Semelhantes}
\subsection{Booking.com}
O site Booking.com permite aos seus utilizadores reservar quartos em hotéis e pensões ao nível mundial, nos com o AirBnB queremos que os nosso utilizadores tenham a possibilidade de arrendar casas inteiras familiares/férias ou quartos para férias, trabalho, ou até alugueres de longa duração, permitido também se desejado que os nossos utilizadores sejam os arrendatários. 
\\
\\
\includegraphics[scale=0.3]{booking.png}
\\
\newpage
\subsection{Trip Advisor}
O Trip Advisor é um site de viagens que fornece informações e opiniões de conteúdos relacionados ao turismo. A informação neste site é maioritariamente gerada pelo utilizador e é sustentado por um modelo de negócio de publicidade.
Através deste site os utilizadores podem relatar as suas experiências em determinados estabelecimentos de alojamento e restauração, podem deixar comentários, fotos e podem classificar (com recurso a estrelas) os mesmos.
Espelhamo-nos nestas funcionalidades para enriquecer o site da AirBnB permitindo aos utilizadores comunicar directamente com a pessoa a quem estão a alugar a casa/quarto e vice versa.
\\
\\
\includegraphics[scale=0.25]{tripadvisor.png}
\\
\newpage
\section{Modelo}
O Kanban é um modelo ágil, criado na década de 60 pela empresa automobilística japonesa Toyota.
Foi o modelo escolhido para implementação do nosso projecto pois mostrou-se um modelo intuitivo e simples de compreender, tendo como um dos diferenciais a visibilidade das tarefas visto que as divide em três etapas a saber, “To do”, “Doing” e “Done”, respectivamente “Tarefas por fazer”, “Tarefas a fazer” e “Tarefas feitas ou finalizadas”. Estas etapas podem ser colocadas em um quadro, o quadro de kanban, permitindo a equipe ter uma visão organizada do projecto possibilitando a selecção das prioridades e remoção das desnecessárias.
\\
\\
\includegraphics[scale=0.3]{kanban.png}


\newpage
\section{Ferramentas Escolhidas}
\subsection{Miro}
O Miro é uma plataforma online de \textit{whiteboard} ou em Português de quadro branco que permite a equipas colaborarem e partilharem ideias entre si, para fazer 
\textit{brainstorming} ou projectos já existentes.

\subsection{Microsoft Teams}
Escolhemos o \textit{Microsoft Teams} como ferramenta de comunicação entre a nossa equipa, pois este permite fazer reuniões virtual e tem boa integração com os nossos email de trabalho, permitindo assim receber todos os alertas mesmo não estando a usar a aplicação nesse preciso momento.
Permite também realizar partilha de ficheiros e partilha de ecrã nas reuniões que são funcionalidades que dão muito jeito à nossa equipa.

\subsection{GitHub}
Para controlo de versões utilizaremos o sistema \textit{Git}, mais especificamente o \textit{GitHub}, que nos vais permitir tem completo controlo do sistema de versões, o que é muito útil pois nos pode permitir reverter ao estado inicial caso tenham ocorrido erros na versão mais recente.
Permite também que todos os membros da equipa tenham sempre a versão mais recente de tudo o que se está a usar.

\subsection{Overleaf}
Iremos utilizar o \textit{Overleaf} para elaboração dos relatórios deste projecto, o \textit{Overleaf} é um editor de \textit{LaTeX} online que além de permitir utilizar todas a funcionalidades da linguagem \textit{LaTeX} permite ainda colaboração simultânea entre 2 ou mais pessoas.

\subsection{Visual Paradigm}
O \textit{Visual Paradigm} é um programa que nos vai permitir utilizar ou fazer qualquer tipo de diagrama \textit{UML} algo que é essencial ao nosso projecto. 
\newpage

\section{Fase de Análise}
\subsection{Recolha de Informação}
 - Nesta fase será feito o levantamento da informação relativa ao projecto, 
 - Prazo: 2 Semanas
    
\subsection{Analise da informação e identificação dos requisitos}
- Cenas da engenharia de requisitos (funcionais, não funcionais e técnicos)
- Prazo: 2,5 Semanas

\subsection{Elaboração do diagrama dos casos de uso}
- Cena muito detalhada de cenas detalhadas do projecto em principio se calhar
- Explicar os requisitos (explicar no geral)
- Prazo: 2 Semanas

\section{Reuniões}
Iremos fazer cerca de duas a três reuniões semanais, sendo que 2 serão via \textit{Microsoft Teams} e a terceira será pessoalmente.

\newpage
\section{Conclusão}

\section{Webgrafia}
\href{https://pt.wikipedia.org/wiki/TripAdvisor}{Wiki do Trip Advisor}\\
\href{https://www.projectbuilder.com.br/blog/quais-sao-os-principais-tipos-de-metodos-ageis/}{Modelo Ageis}
\bibliographystyle{plain}
\bibliography{references}

%\textit{} - italico 
%\textbf{} - bold 

\end{document}