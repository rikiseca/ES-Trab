\documentclass[a4paper]{article}
%\usepackage[utf8]{inputenc}
%\usepackage[english]{babel}
\usepackage[main=portuguese,english]{babel}
\usepackage[utf8x]{inputenc}
\usepackage[T1]{fontenc}
%\usepackage{datetime}
\usepackage[ddmmyyyy]{datetime}
\usepackage{array}
\usepackage{xcolor}
\usepackage{listings}
\usepackage{verbatim}
\usepackage[a4paper]{geometry}
\usepackage{amsmath}
\usepackage{amssymb}
\usepackage{float}
\usepackage{natbib}
\usepackage{graphicx}
\usepackage{textcomp}
\usepackage[colorinlistoftodos]{todonotes}
\usepackage[colorlinks=true, allcolors=blue]{hyperref}
%\usepackage[colorlinks=true, linkcolor=blue, urlcolor=red]{hyperref}

\lstset{basicstyle=\ttfamily, showstringspaces=false, commentstyle=\color{red}, keywordstyle=\color{blue}}

\title{Questionário Airbnb - ES}
\author{Cleyde Varela - 21684 \\ Ricardo Sequeira - 21905}
\date{May 2022}

\begin{document}

\maketitle

\section{Perguntas}

\subsection{ Como é efetuado o registo no site?}
O registo no site pode ser feito de várias maneiras, através de apenas um numero de telefone, através de criação de uma conta com um enderenço de email, ou login através da conta de facebook, google ou apple.

\subsection{Que dados são pedidos aquando da criação da conta?}
É pedido ao utilizador, o primeiro e ultimo nome, data de nascimento, e no caso de não estar a entrar com uma conta existente de outro serviço (google, apple, facebook) que crie uma password.

\subsection{Depois de inseridos esses dados é necessário mais algumas informações?}
Sim depois de inserido esses dados é necessário concordar com a politica de privacidade do site, e inserir um número de telefone válido, e pedida uma foto para anexar ao perfil, estes dados são facultativos na criação de conta pelo que pode ser adicionados mais tarde, o número de telefone torna-se obrigatório no caso de querermos alugar um alojamento para que o arrendatário tenha algum modo de contacto offline.

\subsection{Como funciona o processo de alugar um alojamento?}
Primeiro começamos com o procura do sitio desejando, a barra de pesquisa do airbnb permite refinar a pesquisa selecionando, a localização, a data de check in e de check out e quantos hospedes ficarão hospedados.
Depois de feita a pesquisa com esses termos e escolhido a habitação onde desejamos ficar hospedados e depois de verificar os preços e as caraterísticas da habitação, é só clicar em reservar.
Depois somos levado para uma página onde temos que inserir o numero de telefone (caso não tinha sido inserido aquando do registo inicial) e inserir os dados de pagamento.

\subsection{Que tipos de pagamento são aceites ?}
São aceite pagamentos com cartão de crédito (visa, amex, mastercard) bem como paypal e google pay.

\subsection{O pagamento é feito depois na integra ou permitem pagamentos parcelados?}
Dependendo do custo do alojamento pode ser permitido pagar uma parte aquando da reserva e o restante uma semana antes da data de chegada, a percentagem do pagamento adiantado varia de alojamento para alojamento pode tanto ser 20\% aquando do pagamento, como pode ser 60\%, alojamentos com valores mais baixos apenas oferecem pagamento na integra aquando da reserva.

\subsection{E como funciona no caso de cancelarmos a estadia?}
As politicas de cancelamento são diferentes de alojamento para alojamento, podem ter cancelamento como devolução de 100\% do custo 48 horas antes do dia de chegada, a partir desse dia 50\% de custo menos a taxa de serviço, ou então em certos alojamento podemos pagar um taxa extra para obter as mesmas condições ditas anteriormente.

\subsection{E se for necessário comunicar alguma coisa ao hospede como podemos faze-lo?}
A comunicação com o anfitrião pode ser feita de varias maneiras, ambos o site e a aplicação do airbnb possuem um chat direto com o anfitrião logo após ter sido feita a reserva, ficamos também com acesso ao número de telefone do anfitrião e temos a possibilidade de lhe pedir o email ou qualquer outro meio de comunicação que achemos mais conveniente (que poderá ou não ser disponibilizado dependendo da boa vontade do anfitrião).

\subsection{E quando chegar o dia do check-in qual é o processo seguido para obter aceso ao alojamento?}
O processo varia de anfitrião para anfitrião, mas normalmente antes do dia do check-in chegar é acordada uma hora e uma localização (a do alojamento normalmente), sendo o procedimento o mesmo no check-out.
Pode também acontecer caso o alojamento ser num prédio com porteiro a chave ser deixado com o mesmo pelo anfitrião.

\end{document}
